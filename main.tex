% !TEX TS-program = xelatex

% This is a simple template for a LaTeX document using the "article" class.
% See "book", "report", "letter" for other types of document.

\documentclass[11pt]{article} % use larger type; default would be 10pt

%%% Examples of Article customizations
% These packages are optional, depending whether you want the features they provide.
% See the LaTeX Companion or other references for full information.

%%% PAGE DIMENSIONS
\usepackage{geometry} % to change the page dimensions
\geometry{a4paper} % or letterpaper (US) or a5paper or....
% \geometry{margin=2in} % for example, change the margins to 2 inches all round
% \geometry{landscape} % set up the page for landscape
%   read geometry.pdf for detailed page layout information

\usepackage{graphicx} % support the \includegraphics command and options

\usepackage[parfill]{parskip} % Activate to begin paragraphs with an empty line rather than an indent

%%% PACKAGES
\usepackage{booktabs} % for much better looking tables
\usepackage{array} % for better arrays (eg matrices) in maths
\usepackage{paralist} % very flexible & customisable lists (eg. enumerate/itemize, etc.)
\usepackage{verbatim} % adds environment for commenting out blocks of text & for better verbatim
\usepackage{subfig} % make it possible to include more than one captioned figure/table in a single float
% These packages are all incorporated in the memoir class to one degree or another...

%%% HEADERS & FOOTERS
\usepackage{fancyhdr} % This should be set AFTER setting up the page geometry
\pagestyle{fancy} % options: empty , plain , fancy
\renewcommand{\headrulewidth}{0pt} % customise the layout...
\lhead{}\chead{}\rhead{}
\lfoot{}\cfoot{\thepage}\rfoot{}

%%% SECTION TITLE APPEARANCE
%\usepackage{sectsty}
%\allsectionsfont{\sffamily\mdseries\upshape} % (See the fntguide.pdf for font help)
% (This matches ConTeXt defaults)

%%% ToC (table of contents) APPEARANCE
\usepackage[nottoc,notlof,notlot]{tocbibind} % Put the bibliography in the ToC
\usepackage[titles,subfigure]{tocloft} % Alter the style of the Table of Contents
\renewcommand{\cftsecfont}{\rmfamily\mdseries\upshape}
\renewcommand{\cftsecpagefont}{\rmfamily\mdseries\upshape} % No bold!
\setcounter{tocdepth}{2} % hide subsubsections

%%% END Article customizations

% Hyperlinks, URLs etc.
\usepackage{hyperref}
\usepackage{url}
\hypersetup{
    colorlinks=true,
    citecolor=black,
    urlcolor=black,
    linkcolor=black,
    pagecolor=black,
    anchorcolor=black
}

\newcommand{\movedesc}[3]{
\subsubsection{#1}
\label{sec:#1}
When you \textbf{#2}, #3}

\newcommand{\linkmove}[1]{\hyperref[sec:#1]{\textbf{#1}}}

\newcommand{\rollres}[2]{
\begin{itemize}
\item On a 10+, #1
\item On a 7--9, #2
\end{itemize}}

\newcommand{\rollresfail}[3]{
\begin{itemize}
\item On a 10+, #1
\item On a 7--9, #2
\item On a 6 or less, #3
\end{itemize}}

\newcommand{\movefail}[6]{
\movedesc{#1}{#2}{Roll+#3}
\rollresfail{#4}{#5}{#6}
}

\newcommand{\move}[5]{
\movedesc{#1}{#2}{Roll+#3}
\rollres{#4}{#5}
}

\usepackage{fontspec}
\usepackage[fallback]{xeCJK}
%\setmainfont{PT Serif}
\setCJKmainfont{HanaMinA}

\newcommand{\money}{万}

\usepackage{dialogue}

%%% The "real" document content comes below...

\title{\scshape Mecha World\\
	\small\bfseries A PtbA RP System---with Mecha!}
\author{by Jake Humphrey}
\date{} % Activate to display a given date or no date (if empty),
         % otherwise the current date is printed 

\begin{document}
\maketitle

\tableofcontents

\section{Introduction}
\subsection{Foreword}
This document details a Dungeon World-like tabletop roleplay system for the mecha genre. The creation of this system was motivated by a number of problems faced while playing Mekton Zeta, notably:
\begin{itemize}
\item The amount of time and work required to create a mecha.
\item Lack of roleplaying incentives
\item Overwhelming rule complexity.
\end{itemize}

Conversely, when I played Dungeon World, I was impressed by its ability to integrate the fiction and roleplaying into the mechanics. I want to bring this playstyle to my Mekton Zeta campaign, and whilst creating a whole new system rather than being a better GM may seem like overkill, it will certainly be fun.

\subsection{Influences}
\begin{itemize}
\item Kakumeiki Valvrave
\item Code Geass
\item Tengen Toppa Gurren Lagann
\item Front Mission
\item Hitman
\item Metal Gear
\end{itemize}
\section{Stats}
The base of any RP is the stats. I wanted to cut a fine compromise between the simplicity of, say, Apocolypse World's 5: Cool, Hard, Hot, Smart, and Weird, while still getting the coverage of Mekton Zeta's 9. With that in mind, Mecha World uses the following stats:

\subsection{Cool}
With a ``Cool'' stat existing in both Apocolypse World and Mekton Zeta, it seemed silly to not include it here. Cool symbolises how a character deals with preassure. A Cool character walks out into bulletstorms fearlessly, and take the vilest insults without batting an eye. An Uncool character can most often be found cowering in fear, being reduced to tears, or not getting in the goddamn robot, Shinji.
 
\subsection{Talk}
You may walk the walk, but can you do this? Talk represents how well a character can communicate, whether that be rousing speeches, witty comebacks, or pep talks. A Talkative character can inspire the masses to their cause, humiliate opponents in battles of wits, or brief an entire squadron. An untalkative character misses key details in their explanations, accidentally insults people they're trying to inspire, or thinks of the perfect retort five days later in the shower.

\subsection{Hard}
Another stat stolen from Apocolypse World, Hard represents a character's strength and build. Can your character karate chop a pile of bricks clean in two? Does he eat raw eggs for breakfast, and cow pie for lunch? Does he have to buy two seats at the cinema? Then they're probably a Hard character. If your character eats cup ramen three times a day, faints when struck by falling sunbeams, and doesn't know their hack squats from their hyper extensions, they probably have a low score in this stat.

\subsection{Fast}
The counterpart to Hard among the physical stats, Fast represents a character's running speed and manual dexterity. A fast character may be an Olympic athelete, a ninjutsu master, or even a world-record holding Rubik's cube solver. Slow characters have trouble getting out of bed in the morning, have trouble keeping their balance or footing, and react to dangerous situations a little too late.

\subsection{Tech}
In a darkened room, the curtains drawn, green text adorns a black background of a glowing computer screen. Such is the lifestyle of your typical Techie character. If you want to fix or augment your mecha with cool prototype weapons, hack your opponent's automated defenses to take out their base without getting out of your comfy chair, or create a GUI interface in Visual Basic to back-trace a caller's IP address, choose a high score in this stat. Non-technical characters will be relegated to padded cockpits with single buttons, lest they accidentally set off the alarms whilst trying to force a door open, or cut the wrong wire attempting to defuse a homemade bomb.

\subsection{Plan}
Have you ever walked into a room and forgotten what you came in for? Is your cockpit's monitor adorned with post-its of your various passwords? Do you often find yourself unable to react when the floor is blown out from under your feet by improvised explosive devices? If so, you probably don't have a high Plan score. Planners act fast, and they act with precision. They can make a squadron of 4 soldiers seem like a dozen to the enemy, deduce the keycode to a locked door by the pattern of grime on the keys, or reveal to their enemies that all their weaponry has been disabled from the start. They set up the dominoes, then knock them all down like a house of cards. Checkmate.

\section{Moves}
The moves are the blood of a *-World game, and they keep the game, well, moving. Any move consists of a trigger, and an action. A success generally means the player gets their desired outcome, a partial success is still a success, but a slightly worse overall outcome, and a failure means the GM can make a move.

Class- and Mecha-specific moves will be treated in their own sections, so the basic moves accessible to everyone (human or giant robot) will be detailed here.

\subsection{Basic Moves}

\move{Brawl}{trade blows or blades with someone}{Hard}
{deal your damage to the enemy and choose one:
\begin{itemize}
\item Force the enemy into a nearby danger
\item Avoid a counterattack
\end{itemize}}
{deal your damage to the enemy but receive a counterattack}
{you miss and your enemy takes advantage}

\move{Flourish}{execute a fancy combat technique}{Fast}
{deal your damage to the enemy and avoid their attack}
{choose one:
\begin{itemize}
\item Deal damage to your enemy
\item Avoid a counterattack
\end{itemize}}
{you fail and your enemy takes advantage}

\move{Burst}{spray and pray with an automatic weapon}{Cool}
{deal your damage to the enemy}
{choose two:
\begin{itemize}
\item Deal damage to your enemy
\item Don't hit any unintended targets
\item Still have ammo in the gun
\end{itemize}}
{choose one}

\movenofail{Aim}{take aim and fire with a semi-auto or single-action weapon}{Plan}
{deal your damage to the enemy}
{deal your damage but choose one:
\begin{itemize}
\item You take what you can get: $-1$-harm
\item You have to displace to get a good shot
\item You draw unwanted attention
\end{itemize}}

\move{Hold Fast}{stand your ground}{Cool} % alternate name: Suffer
{you maintain the status quo despite outside forces}
{choose two:
\begin{itemize}
\item maintain the status quo
\item you don't take attrition
\item you don't lose something important
\end{itemize}}
{choose one}

\move{Taunt}{make a snarky comment at someone}{Talk}
{choose two:
\begin{itemize}
\item you provoke them into making an error
\item you provoke them into revealing information
\item you don't provoke a retaliation
\end{itemize}}
{choose one}
{your target ignores your foolish yapping}

\movedesc{Defy}{make a last-ditch attempt to avoid a calamity}{pick the stat you are using.}
If you do it:
\begin{itemize}
\item By ignoring the danger, use Cool
\item With quick talking, an inspiring word, or a witty retort, use Talk
\item With a feat of strength or endurance, use Hard
\item With lightning reflexes or speed, use Fast
\item By using gadgets or technical knowledge, use Tech
\item With a premeditated strategy or a bright idea, use Plan
\end{itemize}
Assuming you have time to react, Roll+Stat:
\rollresnofail{you do what you set out to, the threat doesn’t come to bear}
{you stumble, hesitate, or flinch: the GM will offer you a worse outcome, hard bargain, or ugly choice}

\move{Jury Rig}{use a piece of equipment for something other than its intended purpose}{Tech}
{you succeed}
{you succeed, but choose one:
\begin{itemize}
\item you damage the equipment beyond repair
\item you set off a previously-unknown function of the equipment
\item you hurt yourself in the process
\end{itemize}}
{just choose one}

\subsection{Special Moves}

\movedesc{Encumbrance}{make a move while carrying weight}{you may be encumbered. If your weight carried is:}
\begin{itemize}
\item Equal to or less than your load, you suffer no penalty
\item Less than or equal to your load+2, you take $-1$ ongoing until you lighten your burden
\item Greater than your load+2, you have a choice: drop at least 1 weight and roll at $-1$, or automatically fail
\end{itemize}

\move{Source Parts}{browse a store's catalogue for mundane mecha parts, personal items, or repairs}{Talk}
{they have what you need, and you can get a good deal}
{they have what you need, for about what you expected}
{they have what you need, but it costs more than you thought it would}

\move{Rare Parts}{make a request to a store for rare mecha parts}{Talk and describe what you're looking for}
{they happen to have it in, or the nearest thing that exists, and you can have it for the right price}
{they have it, or can get it, but there's a catch. The GM will choose one:
\begin{itemize}
\item It costs a lot.
\item It's not quite what you wanted.
\item It comes with strings attached, or a favour owed.
\end{itemize}}
{they don't have it, sorry.}
\section{Classes}

\subsection{Operator}
{\itshape You're not like the others, are you? They all fight for something, or maybe it's more accurate to say they all still have something worth fighting for. What do you have? A klunky piece-of-shit gun, battered body armour, and a hell of a lot of history. 

How many people have you killed? Do you still remember each of their faces? Of course you fucking don't. Whatever humanity you had went out the window a long time ago. You're little more than machine now. Even the mecha are more human than you are.}

\subsubsection{Look}

Choose one for each:

Dead Eyes, Hard Eyes, or Gleeful Eyes

Military Gear, Concealing Clothes, or Casual Wear

\subsubsection{Stats}
Choose one set:
\begin{itemize}
\setlength\itemsep{0em}
\item Cool+2 Talk-2 Hard=0 Fast=0 Tech+1 Plan+2
\item Cool+2 Talk-1 Hard+1 Fast=0 Tech+1 Plan=0
\item Cool+2 Talk-1 Hard=0 Fast+1 Tech=0 Plan+1
\item Cool+2 Talk=0 Hard+1 Fast=0 Tech-1 Plan+1
\end{itemize}

\move{Volley}{target multiple enemies with a single burst}{Cool}
{Deal your damage to each enemy}
{Deal your damage to half the enemies, rounded down. The GM will tell you which ones}
{as above, but you also may hit unintended targets}

\move{Deathwish}{walk out into a storm of bullets}{Cool}
{Choose up to 3 visible enemies in range and make immediate consecutive attacks against them}
{Choose up to 3 enemies and make immediate consecutive attacks against them at the cost of taking 1-harm per target chosen}
{Make an attack against 1 enemy but receive a counterattack}

\move{Breach and Clear}{make an explosive entry into a hostile area}{Plan}
{Choose up to 3 visible enemies in range and make immediate consecutive attacks against them}
{Choose up to 2 enemies, but 1 from the following list}
{Choose one enemy and two from the list:
\begin{itemize}
\item Flying debris inflicts 1-harm with 1 piercing to you
\item There are more enemies than expected
\item You hit an unintended target
\end{itemize}}

\subsubsection{Equipment}
You have an SMG (2-harm, close, full-auto, loud, 2 weight), choose its 2 attachments:
\begin{itemize}
\item Silencer ($-$loud)
\item Skeleton Stock ($-1$ Weight)
\item AP ammo (+1 piercing)
\item Collapsible Stock (+hand)
\item IR scope (night vision)
\end{itemize}

You also have a pistol (2-harm, close, hand, semi-auto, loud, 1 weight), choose its attachment:
\begin{itemize}
\item Silencer ($-$loud)
\item .45 cal (+1-harm)
\item AP ammo (+1 piercing)
\end{itemize}

You get:
\begin{itemize}
\item Bulletproof Vest (1 armour, 1 weight)
\item 2 Stimpaks ($-1$-harm, 1 Weight)
\item 2 Explosive Kits (2 weight)
\item 3 Tactical Grenades (0 weight)
\end{itemize}



\subsection{Wiseguy}
{\itshape Where would these misfits be without you, eh? Just the other day you tried to strike up a rousing conversation with the Brawler, and he \emph{hissed} at you. Hissed! Like a cat! No, no, no, these guys would be mincemeat by now if you weren't around to smooth things over.

Leave the deal-brokering, the negotiations, and the sweet-talking to you. You'll always be around to talk your way out of a sticky situation. No need to thank you, it's all part of the service. Well, unless they're armed, in which case your former friends are on their own!}
\subsubsection{Look}
Choose one for each:

Shifty Eyes, Sharp Eyes, or Gleeful Eyes

Tidy Hair, Slick Hair, or Ponytail

Concealing Clothes, Business Wear, or Snazzy Clothes

\subsubsection{Stats}
Choose one set:
\begin{itemize}
\setlength\itemsep{0em}
\item Cool+1 Talk+2 Hard-1 Fast=0 Tech=0 Plan+1
\item Cool=0 Talk+2 Hard-1 Fast+1 Tech+1 Plan=0
\item Cool-1 Talk+2 Hard-2 Fast+2 Tech+1 Plan+1
\item Cool=0 Talk+2 Hard-1 Fast-1 Tech=0 Plan+2
\end{itemize}

% written by Luke
\move{Networking}{arrive at a new location and mention an old accomplice or contact}{Talk, describe how you remember them, and recount a whacky anecdote about them}
{they're in town, and owe you a favour or their life (take +1 ongoing to interacting with them until you're even)}
{they were living here at least recently, at the GM's discretion maybe they're still here; at the very least the trail will be hot. However, things have changed since the good old days---pick 1:
\begin{itemize}
\item They've changed.
\item You've changed.
\item They don't remember you, at least not fondly.
\item You still owe them a favour or money
\end{itemize}}
{they're long gone or at the GM's discretion they're in town, but the winds of fate have positioned them to oppose you, not aid you.}

\move{Wisecrack}{tell a joke to defuse a tense situation}{Talk}
{Everyone present relaxes and becomes a little more amicable}
{Your joke falls flat, but the attempt causes everyone to calm down. Everyone becomes more amicable, but you take $-1$ forward, in shame.}
{Your light-hearted treatment of the situation pisses people off.}

\movenofail{Eye On The Door}{are in too deep and need a way out}{Plan, naming your escape route:}
{you're gone. Catch you later, suckers!}
{you can stay or go, but if you go you either leave something behind, or take something with you, the GM will tell you what.}

\subsubsection{Equipment}
You have your trusty piece(s) for when things go hot. Choose 1 and give it a nickname:
\begin{itemize}
\item Gold DE .50 AE (2-harm, 1 piercing, hand, close, semi-auto, valuable, 2 weight)
\item Dual Mini-Uzis (2-harm, two-handed, hand, close, full-auto, 1 weight) 
\end{itemize}



\subsection{Brawler}
{\itshape They always say weapons don't kill people, people do. Well, what are the weapons for, then? A weapon is nothing without a strong body wielding it. Look at the others. They think their pop-guns and glorified tree branches will protect them in a fight. Maybe, but where will they be when those weapons are taken away?

They call you names sometimes. ``Meathead'', they say. ``Numbskull'', ``Shit-For-Brains''. They laugh. Well, they'll be laughing on the other side of their faces one day. When the enemy has them jailed, and they see you removing the bars one by one like so many toothpicks. Or maybe you'll just crush one of their skulls like a watermelon. Nobody will be laughing then.}

\subsubsection{Look}
Choose one for each:

Wild Eyes, Determined Eyes, or Bored Eyes

Shorn Hair, Wild Hair, or Mask

Sporty Gear, Outdoor Gear, or Remnants of Experimentation

\subsubsection{Stats}
Choose one set:
\begin{itemize}
\setlength\itemsep{0em}
\item Cool+2 Talk=0 Hard+2 Fast+1 Tech-2 Plan=0
\item Cool+1 Talk+1 Hard+2 Fast+1 Tech-1 Plan-1
\item Cool-1 Talk-1 Hard+2 Fast+2 Tech+1 Plan-1
\item Cool+1 Talk+1 Hard+2 Fast+1 Tech+1 Plan-2
\end{itemize}

\subsubsection{My Guns Are Right Here}
Your unarmed strike counts as a weapon: (2-harm, two-handed, intimate, fused, 0 weight).

\movedesc{Full Force}{succeed a Brawl with a 10+}{you may choose both options.}

\move{Imagine This Is Your Head}{intimidate information out of an enemy with a show of force}{Hard}
{they immediately tell you what you want to know. What you do next is up to you.}
{they agree to tell if you let them go, and want to be certain they won't be harmed.}
{they spit in your face, to the extent possible.}

\movedesc{It's a Monster!}{stand your ground}{you may roll Hold Fast with Hard instead of Cool}

\subsubsection{Equipment}
You have your fists, but I guess you can also have a gun, you know, just to be on the safe side. Choose one:
\begin{itemize}
\item Hunting Shotgun (3-harm, semi-auto, loud, messy, close, 2 weight)
\item Grenade Launcher (4-harm, semi-auto, loud, messy, close, area, 3 weight)
\item Rocket Launcher (4-harm, single, loud, messy, close, far, area, 3 weight)
\end{itemize}

You get:
\begin{itemize}
\item 2 Stimpaks ($-2$-harm, 1 weight)
\end{itemize}



\subsection{Swordsmaster}
{\itshape These fools lack honour and discipline. Look at them now, laughing carelessly over a shared lunch. They could be attacked right this minute, and only you would be able to defend yourself. While they waste away their time you sit here, tirelessly drawing and resheathing your sword to shave off those precious milliseconds that could mean the difference between life and death.

A warrior's life is tough, to be sure, but it has its rewards. What are they to you? To lay down your life in defense of your lord, like the samurai of old? Or do you prefer to feel the breath of your enemy leave them, while you bask in the knowledge that when it came right down to it, you were better than them? Whatever your goal, you will find it hard to attain with these weaklings around. They lack honour and discipline, yes, but you can change that.}

\subsubsection{Look}

Choose one for each:

Tired Eyes, Fierce Eyes, or Piercing Eyes.

Topknot, Wild Hair, or Shorn Hair

Gi and Hakama, Noble Wear, or Bandages

\subsubsection{Stats}
Choose one set:
\begin{itemize}
\setlength\itemsep{0em}
\item Cool+1 Talk-1 Hard+1 Fast+2 Tech-1 Plan+1
\item Cool+1 Talk+1 Hard+1 Fast+2 Tech-1 Plan-1
\item Cool=0 Talk-1 Hard+2 Fast+2 Tech-1 Plan-1
\item Cool+1 Talk-1 Hard=0 Fast+2 Tech+1 Plan=0
\end{itemize}

\subsubsection{Signature Weapon}
You are the bone of your sword. Perhaps it was created by a master smith centuries ago, or perhaps you found it on the floor. Regardless, this is your sword and it is irreplaceable, so don't lose it.

Your base sword comes with a matching sheath and is (2-harm, hand, 2 weight).

Choose a look:
\begin{itemize}
\item Straight or Curved
\item Single-edged, Double-edged, or Thrusting
\item Simple Design, Elaborate Design, or High-Tech Design
\end{itemize}

Choose two enhancements:
\begin{itemize}
\item Serrated (+1-harm)
\item Vibrating (+1 piercing, messy, hi-tech)
\item Disposable (infinite)
\item Perfectly-Balanced (precise)
\item Tasseled (+short, thrown)
\item Well-Crafted ($-1$ weight)
\item Twin (you have two and can dual-wield, but each individually is 1-harm and 1 weight. You can deal damage with both during an attack, but your other enhancement applies to only one blade)
\end{itemize}

\move{Quickdraw}{are attacked in melee by an enemy while your sword is sheathed}{Fast}
{draw your sword and inflict your damage, negating the incoming attack}
{draw your sword and inflict your damage, but the enemy's attack still lands}
{the enemy reaches you before you can draw your sword}

\movenofail{You Are Already Dead}{resheathe your sword immediately after dealing damage}{Cool}
{Choose 2:
\begin{itemize}
\item Additional wounds are revealed on your enemy, adding 1-harm to the damage and applying the messy tag.
\item Your spirit overflows, hitting nearby enemies for 1s-harm
\item A nearby mook flees in terror
\item A nearby character is impressed
\end{itemize}}
{choose 1}

\subsubsection{Equipment}
You get your signature weapon, but choose your defenses:
\begin{itemize}
\item Padded Clothes (1-armour, 1 weight)
\item Samurai Armour (2-armour, 3 weight)
\item Exoskeleton (2-armour, fused, 0 weight)
\end{itemize}



\subsection{Hikikomori}
{\itshape What was that noise? Is someone there? Maybe if you're quiet enough, they'll think you're not in. You shun outside contact. Friends only ever let you down. To support yourself, here, with your own two hands is enough. Well, in addition to the food your nice older sibling leaves outside your door.

The gentle hum of the machinery, the moonlight streaming in through the viewing port. This is the life. Solitude is not loneliness; that's what the others don't understand. Always asking what's wrong. They'll never understand you. Not like you do.}

\subsubsection{Look}
Choose one for each:

Tired Eyes, Calculating Eyes, or Hair Over Eyes.

Messy Hair, Long Straight Hair, or Covered Hair

Tracksuit, T-shirt and Jeans, or Pyjamas

\subsubsection{Stats}
Choose one set:
\begin{itemize}
\setlength\itemsep{0em}
\item Cool+1 Talk-2 Hard-1 Fast+1 Tech+2 Plan+2
\item Cool-1 Talk+1 Hard=0 Fast=0 Tech+2 Plan+1
\item Cool+1 Talk-1 Hard+2 Fast-1 Tech+2 Plan-1
\item Cool=0 Talk=0 Hard-2 Fast+2 Tech+2 Plan+1
\end{itemize}

\move{Security Expert}{attempt to overcome network security}{Tech}
{choose three from the list below:
\begin{itemize}
\item You learn something new about your adversary
\item You disable something of the enemy's
\item You sow misinformation among the enemy
\item Your intrusion is undetected
\end{itemize}}
{choose one}
{you are detected before you can do anything}

\move{Don't Come In!}{trap a room or entrance}{Plan and deplete 1 Tech Item}
{hold 3}
{hold 2}
{hold 1, plus whatever the GM says when you spend it}
Spend your hold when a character approaches the trap to choose one of the following:
\begin{itemize}
\item neutralise the intruder
\item observe the intruder
\item play a prerecorded message to the intruder
\end{itemize}

\movedesc{Kindred Spirits}{talk with another weirdo}{invert your talk score when rolling +Talk with them}

\subsubsection{Equipment}
Choose a weapon:
\begin{itemize}
\item Antique sword (hand, valuable, 2-harm, 2 weight)
\item Taser (intimate, 2s-harm, hi-tech, 0 weight)
\end{itemize}

You get:
\begin{itemize}
\item 2 Tech Items (hi-tech, 1 weight)
\item Laptop (valuable, hi-tech, 1 weight)
\item An item which affords you emotional security, at your option it can also provide 1-armour
\end{itemize}



\subsection{Assassin}
{\itshape For what reason do you do this? Don't bother lying to me, everyone has a reason. People may fight trivially, but they always kill for a cause. Be that a higher power or ideology, the sweet taste of revenge, or just the undeniable allure of filthy lucre.

You have as myriad methods of killing as you have targets. The partygoing billionaire may warrant a .50 cal bullet to the brain from the roof of an adjacent building, but the paranoid mob boss will require a more\ldots intimate approach. Do you feel pleasure as the garotte denies vital oxygen to his lungs? Did this man bring untold suffering to countless innocents? Or was he just on the wrong side of a million-dollar transaction?}

\subsubsection{Look}
Choose one for each:

Dead Eyes, Calculating Eyes, or Hard Eyes.

Shorn Hair, Slick Hair, or Hooded

Fatigues, Formal Wear, or Concealing Wear

\subsubsection{Stats}
Choose one set:
\begin{itemize}
\setlength\itemsep{0em}
\item Cool+1 Talk=0 Hard+1 Fast=0 Tech-1 Plan+2
\item Cool=0 Talk+1 Hard=0 Fast+1 Tech-1 Plan+2
\item Cool+1 Talk-2 Hard-1 Fast+2 Tech+1 Plan+2
\item Cool+2 Talk-1 Hard-1 Fast-1 Tech+1 Plan+2
\end{itemize}

\move{Assassinate}{try to take down a target with a high-powered rifle}{Plan}
{you take out your target without leaving evidence. At your option you can also make it look like an accident}
{Choose one:
\begin{itemize}
\item you take out your target but leave behind evidence
\item your target is permanently crippled in some way, but lives. However, you do not leave behind evidence
\end{itemize}}
{your target is only permanently crippled, and you leave evidence}

\move{Selective Hearing}{attempt to pick out a target in a crowd that you're in}{Cool and say what you're listening out for}
{you identify your target and also overhear some interesting information}
{you identify your target, but they're just leaving}
{you narrow your target to one of three people, all of whom are heading in different directions}

\movedesc{Silent Takedown}{catch an enemy unawares or have an enemy defenseless}{you can kill them without a sound, you don't need to use a weapon. Snap their neck if you have to.}

\subsubsection{Equipment}

You have your trusty sniper rifle (harm-3, far, single, loud, hi-tech, 2 weight). Choose 2 attachments:
\begin{itemize}
\item Silencer ($-$loud)
\item ACOG scope (+close)
\item .50 cal ammo (ignores armour, messy)
\item Bipod (+distant)
\item Magazine feed ($-$single, +semi-auto)
\end{itemize}

You also pick one of the following sidearms:
\begin{itemize}
\item Silenced 9mm (2-harm, hand, close, semi-auto, 1 weight)
\item Sawn-off Shotgun (3-harm, close, loud, 2 weight)
\end{itemize}

You get:
\begin{itemize}
\item Butterfly knife (3-harm, intimate, 0 weight)
\item Garrote wire
\item Leather gloves
\item 1 Explosives Kit (1 weight)
\end{itemize}

%\subsection{Class Name}
%{\itshape}
%\subsubsection{Look}
%\subsubsection{Stats}
%Choose one set:
%\begin{itemize}
%\setlength\itemsep{0em}
%\item Cool=0 Talk=0 Hard=0 Fast=0 Tech=0 Plan=0
%\item Cool=0 Talk=0 Hard=0 Fast=0 Tech=0 Plan=0
%\item Cool=0 Talk=0 Hard=0 Fast=0 Tech=0 Plan=0
%\item Cool=0 Talk=0 Hard=0 Fast=0 Tech=0 Plan=0
%\end{itemize}
%\subsubsection{Equipment}
\section{Mecha}

It wouldn't be Mecha World without mecha: the giant robots our heroes use to fight the bad guys---usually in weaker mecha.

Each class has their own unique mecha weapon and ability, but it's up to the group to decide how these are realised. Has each character had their mecha custom-built for them, or did the party come across them recently, conveniently geared up to match their strengths?

The stats of the mecha themselves reflect the strengths and weaknesses of their pilots, meaning rolls whilst in the mecha use the pilot's stats; even when it wouldn't seem to make sense. You might think, why does it matter how strong the pilot is when he's controlling fifty tonnes of hydraulics? This again can can be reflected in the fiction. Perhaps the mecha have to be closely tuned to the capabilities of the pilot for optimal compatibility, or maybe the mecha are literally powered by the will or energy of the pilots themselves. (Think Spiral Power)

%Mecha are generally humanoid; they have a head, torso, two arms, and two legs. But here the visual similarities between them end. 

\subsection{Mecha Items}
Okay, so you have your mecha and its weapon, but the mecha's load capacity is much higher than that, right? The mecha has the same max load as the pilot, with a few important distinctions.

First off, they're clearly not in the same units. Giant robots can carry objects orders of magnitude heavier than people can. So people can't carry mecha-sized items, and mecha can trivially carry any number of human-sized items, within reason.

Secondly, whilst humans can overexert themselves to carry more then their max load (invoking the Encumbrance move), mecha are a bit more concrete in their abilities. Mecha can not usually carry above their max load.

But don't start calculating how many valuable relics or moon-rocks you can fill up on just yet. You're going to need all the surplus load you can get for Mecha Items. Your mecha's Mecha Items are chosen at character generation. When you stop for repairs, you can replenish your depleted Mecha Items, as well as optionally choose one to replace with a different type. Depleted or shed Mecha Items also weigh nothing, so you can free up space for all those oversized baubles your pilots can't carry.

The available Mecha Items (all 1 weight) are:

\paragraph{Defensive Measures:}
These are things like ablative shielding or anti-lock-on flares: passive items which prevent an attack from dealing damage. When you take damage, you can deplete as many of this item as you want to prevent harm 1-for-1.

\paragraph{Guided Missiles:}
Rather than being a separate weapon, mecha in Mecha World just have racks upon racks of HE missiles or similar packed into them. This item represents one salvo, which can be fired-and-forgotten to seek out the nearest enemy and deal 1-harm to them---provided, of course, that they don't evade the attempt with a Defensive Measure or some other ability.

\paragraph{Impulse Thrusters:}
These rockets expel a sizeable portion of propellant incredibly fast, exerting a considerable force that can be used to propel your mecha in a given direction, or other things away from you, depending on how hard you hold on to the ground. Unfortunately, their warm-up time makes them unsuitable for use in evasive manoeuvres.
\section{Equipment}
\subsection{General Equipment Tags}
\begin{itemize}
\item \textit{Fused}: This item is integrated into the user's body, and can not be removed at will.
\item \textit{Hi-Tech}: This item runs on electricity. As well as being harder to find or repair, EMPs will disable these items. Note that if an enhancement gives this tag, an EMP may not make the item completely useless. A Swordsmaster's vibrating sword, for example, can still be used as a sword, just without the benefits the vibrating enhancement brings.
\item \textit{Loud}: Using this item makes a loud noise that may attract attention. Not for stealth missions.
\item \textit{Night Vision}: This item allows the user to see outside the normal spectrum of visible light, typically IR. As well as offering vision in darkness, this can also penetrate obstacles like smoke grenades. 
\item \textit{Two-Handed}: This item requires two hands to be used effectively. Note this does not always mean the two hands need to be holding the item. A rapier, for example, may require the wielder to use their other hand as a counterbalance. 
\item \textit{Valuable}: This item can be sold for a tidy sum, if you can find a buyer.
\end{itemize}

\subsection{Weapon Tags}
\begin{itemize}
\item \textit{$n$-harm}: This item inflicts $n$ damage to an enemy. If an enemy's damage is larger than their hp, they die.
\item \textit{$n$s-harm}: This item inflicts $n$ stun damage to an enemy. If the sum of an enemy's damage and stun damage is larger than their hp, they pass out.
\item \textit{$n$ piercing}: Damage inflicted by this weapon treats the enemy's armour as $n$ less. 
\item \textit{Ignores Armour}: Damage inflicted by this weapon treats the enemy's armour as 0. 
\item \textit{Messy}: This weapon does damage in a particularly destructive way, ripping people and things apart.
\item \textit{Precise}: Rewards careful strikes. You use Fast to Brawl with this weapon, not Hard.
\item \textit{Versatile}: Affords a higher freedom of movement, for example, a snake sword or three-section staff. You use Hard to Flourish with this weapon, not Fast.
\end{itemize}

\subsubsection{Range Tags}
\begin{itemize}
\item \textit{intimate}: The closest range, weapons with this range can only be used during serious violations of your personal space, such as if you are in a pile on the floor, if you dodge an enemy's melee attack and get in close, or of course if you have them completely at your mercy.
\item \textit{hand}: This range indicates the range of swung weapons, typically swords. Think about 2 paces.
\item \textit{close}: This is the closest ``ranged'' range. Pistols, Shotguns, SMGs and Assault Rifles all use this range.
\item \textit{far}: This is outside the range of most personal firearms. Scoped Rifles mostly fall into this range.
\item \textit{distant}: Enemies at this range are barely visible. A high-powered rifle with stabilisation is required to even attempt a shot at this range. Either that or guided missiles.
\end{itemize}

\subsubsection{Weapon Action Tags}
\begin{itemize}
\item \textit{single}: After firing this weapon a new round must be loaded into the chamber manually.
\item \textit{semi-auto}: After firing this weapon, the chamber is filled automatically, allowing for another consecutive shot.
\item \textit{full-auto}: Holding down the trigger on this weapon releases a burst of fire.
\end{itemize}

\subsection{Armour Tags}
\begin{itemize}
\item \textit{$n$ armour}: Damage recieved while wearing this armour is reduced by $n$
%\item \textit{}: 
\end{itemize}

\end{document}
