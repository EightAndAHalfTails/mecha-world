% Tactician Inspiration: Lelouch, Maou, Light, Sora, 
\subsection{Tactician}
{\itshape How do you see the battlefield? Not just as the chaos of
  gunfire and explosions, but as something more\ldots concerted. You
  see the intention behind every bullet, the desperation behind every
  swing. When you look into the eyes of your enemies, you see their
  plans, motives, and weaknesses laid bare.

When you turn those eyes upon your friends, what do you see? Do you
reduce them in your mind down to a list of strengths and weaknesses,
opportunities and threats? They each have their part to play in your
plan. Let's hope you haven't missed anything.}
\subsubsection{Look}
Choose one for each:
Determined Eyes, Calculating Eyes, or Bored Eyes.

Neat Hair, Messy Hair, or Styled Hair.

Military Garb, Casual Clothes, or Formal Wear

\subsubsection{Stats}
Choose one set:
\begin{itemize}
\setlength\itemsep{0em}
\item Cool${=}0$ Talk$+2$ Hard$-1$ Fast$-1$ Tech${=}0$ Plan$+2$
\item Cool${=}0$ Talk$+1$ Hard${=}0$ Fast$-2$ Tech$+2$ Plan$+2$
\item Cool$+1$ Talk$+1$ Hard$-1$ Fast${=}0$ Tech${=}0$ Plan$+2$
\item Cool$+1$ Talk${=}0$ Hard${=}0$ Fast$-1$ Tech$+1$ Plan$+2$
\end{itemize}

Your Max Health and Max Load are both 4+Hard.

\subsubsection{Philosophy}
Choose one:
\paragraph{Yin:} Kill someone to tie up loose ends.
\paragraph{Yang:} Avoid an unecessary death.

\subsubsection{Introductions}
At the start of the first session, or when a new character joins, go round for introductions.
When you introduce your character, answer the following questions out loud:
\begin{itemize}
\item What do you look like? What impression does that give?
\item What do people think you're like? What about after they talk to you for five minutes?
\item In what way is people's first impression of you wrong or not quite the full story?
\end{itemize}

Then ask at least one of the following of the other characters:
\begin{itemize}
\item "Which one of you evidently distrusts me?" Whoever answers, offer that player 1-interest.
\item "Which one of you do I consider the most capable?" Whoever answers, offer that player 1-interest.
\end{itemize}
Tell the others 0-interest.

After each other character's introduction:
\begin{itemize}
\item Choose one and tell them "I've been studying you closely for some time." Your interest in them is 2 regardless of what they offer you.
\item Choose one and tell them "I consider you unpredictable." Your interest in them is 2 regardless of what they offer you.
\item You’ve done your homework. Your interest in the others is 1 more than whatever they offer you.
\end{itemize}

\subsubsection{Starting Moves}
You start with \linkmove{Bodyguard} and \linkmove{Command}, plus two moves from the others below:

\subsubsection{Bodyguard}\label{sec:Bodyguard}
You have a bodyguard: a loyal protector who will serve you to the death. Name them and give them a one-line description.

Choose a base:
\begin{itemize}
\item Training $+2$, Cunning $+1$, 1 Armour, Humanity $+1$
\item Training $+2$, Cunning $+2$, 0 Armour, Humanity $+1$
\item Training $+1$, Cunning $+2$, 1 Armour, Humanity $+1$
\item Training $+3$, Cunning $+1$, 1 Armour, Humanity $+2$
\end{itemize}

Choose as many strengths as their Training:

\textit{Fast, burly, calm, intimidating, perceptive, stealthy, ferocious.}

Your bodyguard protects you from attackers. Choose as many additional trainings as their Cunning:

\textit{Search, scout, guard, gather information, repair, work for money, teach.}

Choose as many weaknesses as their Humanity:

\textit{Cowardly, insane, scarred, oblivious, ambitious, stubborn, lame.}

\subsubsection{Command}\label{sec:Command}
As well as protecting you from attackers, your bodyguard can also attack targets for you. When you \textbf{use your bodyguard as a weapon}, Roll+Cunning for any moves triggered instead of whatever that move demands. Your bodyguard inflicts harm equal to their Training.

Otherwise, when you work with your bodyguard on something they're trained in:
\begin{itemize}
\item and you attack the same target, add their Training to your damage.
\item and you take damage, add their armour to your armour.
\item and you put your heads together, add their Cunning to any roll you make.
%\item and you Defy, add their cunning to your roll.
\end{itemize}

\movefail{Best-Laid Plans}{start the session}{Plan}{hold 3}{hold 2}{hold 1, plus whatever the GM says when you spend it}

You may spend your hold 1-for-1 at any time during the session to have one of the following effects occur, either because you planned it or by pure luck:
\begin{itemize}
\item One weapon is disabled or broken.
\item One object explodes.
\item You come across a needed item.
\end{itemize}

% Idea: give each character a role, and they get +1 forward when acting in accordance with their role?
\move{Mission Briefing}{explain your plan to the other players}{Talk, describing up to 3 details.}{the plan is foolproof, and fills your comrades with hope: everyone present takes $+1$ forward now, and again when sticking to the plan.}{the plan is sound: everyone present takes $+1$ when sticking to it.}

\movedesc{Attention to Detail}{analyse a situation or try to figure someone out}{you are not restricted to the questions on the list.}

\subsubsection{Move 6 Placeholder} % TODO

\subsubsection{Equipment}
You get:
\begin{itemize}
\item Service Pistol (2-harm, hand, close, semi-auto, loud, 1 weight) 
\item 1 Tech Item (1 weight)
\item 1 Explosives Kit (3 uses, 1 weight)
\item 3 Tactical Grenades (0 weight)
\end{itemize}

\subsubsection{Improvement}
\begin{itemize}
\item get $+1$ Cool (max $+2$)
\item get $+1$ Talk (max $+2$)
\item get $+1$ Hard (max $+2$)
\item get $+1$ Fast (max $+2$)
\item get $+1$ Tech (max $+2$)
\item get $+1$ Plan (max $+3$)
\item get a new Tactician move
\item get a new Tactician move
\item get an Agency, \linkmove{Mission}, and \linkmove{Call For Support}
\item get a Paragon move
\end{itemize}