\section{Equipment}
\subsection{Money}
The money used in the base Mecha World is the \money, pronounced ``man''. It is equivalent to 10,000¥, or about \$100 or £50 at today's exchange rates. Of course, your Mecha World can use barter, bottlecaps or galactic credits, whatever fits the fiction.

\subsection{General Equipment Tags}
\begin{itemize}
\item \textit{Fused}: This item is integrated into the user's body, and can not be removed at will.
\item \textit{Hi-Tech}: This item runs on electricity. As well as being harder to find or repair, EMPs will disable these items. Note that if an enhancement gives this tag, an EMP may not make the item completely useless. A Swordsmaster's vibrating sword, for example, can still be used as a sword, just without the benefits the vibrating enhancement brings.
\item \textit{Loud}: Using this item makes a loud noise that may attract attention. Not for stealth missions.
\item \textit{Night Vision}: This item allows the user to see outside the normal spectrum of visible light, typically IR. As well as offering vision in darkness, this can also penetrate obstacles like smoke grenades. 
\item \textit{Two-Handed}: This item requires two hands to be used effectively. Note this does not always mean the two hands need to be holding the item. A rapier, for example, may require the wielder to use their other hand as a counterbalance. 
\item \textit{Valuable}: This item can be sold for a tidy sum, if you can find a buyer.
\end{itemize}

\subsection{General Equipment List}
\paragraph{Tech Item:} 1\money. The exact specifications of the Tech Item are clarified at use time, but it could be anything from a IR beam sensor to a retinal scanner hacker.
\paragraph{Explosives Kit:} 2\money, 3 uses. Reach into to the Explosives Kit and pull out something explosive. This could contain anything from frag grenades to remote-detonated C4 packs to anti-mecha land mines, but not generally reusable weapons like Rocket Launchers.
\paragraph{Tactical Grenades:} 1\money, 3 uses. Chuck a tactical grenade to do something to your enemy aside from blowing them up. This might be stunning them with a flashbang, releasing a smokescreen, or disabling tech with a burst of EMP.

\subsection{Weapon Tags}
\begin{itemize}
\item \textit{$n$-harm}: This item inflicts $n$ damage to an enemy. If an enemy's damage is larger than their hp, they die.
\item \textit{$n$s-harm}: This item inflicts $n$ stun damage to an enemy. If the sum of an enemy's damage and stun damage is larger than their hp, they pass out.
\item \textit{$n$ piercing}: Damage inflicted by this weapon treats the enemy's armour as $n$ less. 
\item \textit{Area}: This weapon's damage affects all nearby the target. Don't point into a brawl unless you don't like any of them.
\item \textit{Ignores Armour}: Damage inflicted by this weapon treats the enemy's armour as 0. 
\item \textit{Messy}: This weapon does damage in a particularly destructive way, ripping people and things apart.
\item \textit{Precise}: Rewards careful strikes. You use Fast to Brawl with this weapon, not Hard.
\item \textit{Versatile}: Affords a higher freedom of movement, for example, a snake sword or three-section staff. You use Hard to Flourish with this weapon, not Fast.
\end{itemize}

\subsubsection{Range Tags}
\begin{itemize}
\item \textit{intimate}: The closest range, weapons with this range can only be used during serious violations of your personal space, such as if you are in a pile on the floor, if you dodge an enemy's melee attack and get in close, or of course if you have them completely at your mercy.
\item \textit{hand}: This range indicates the range of swung weapons, typically swords. Think about 2 paces.
\item \textit{close}: This is the closest ``ranged'' range. Pistols, Shotguns, SMGs and Assault Rifles all use this range.
\item \textit{far}: This is outside the range of most personal firearms. Scoped Rifles mostly fall into this range.
\item \textit{distant}: Enemies at this range are barely visible. A high-powered rifle with stabilisation is required to even attempt a shot at this range. Either that or guided missiles.
\end{itemize}

\subsubsection{Weapon Action Tags}
\begin{itemize}
\item \textit{single}: After firing this weapon a new round must be loaded into the chamber manually.
\item \textit{semi-auto}: After firing this weapon, the chamber is filled automatically, allowing for another consecutive shot.
\item \textit{full-auto}: Holding down the trigger on this weapon releases a burst of fire.
\end{itemize}

\subsection{Weapon Examples}
\begin{itemize}
\item Duelling Rapier (2-harm, 1 piercing, precise, hand, 1 weight, 1\money)
\item Stun Baton (2s-harm, hand, 2 weight, 2\money)
\item 9mm (2-harm, hand, close, semi-auto, loud, 1 weight, 2\money)
\item Machine Pistol (1-harm, hand, close, full-auto, loud, 1 weight, 2\money)
\item SMG (2-harm, hand, close, full-auto, loud, two-handed, 2 weight, 3\money)
\item Assault Rifle (2-harm, close, semi-auto, full-auto, loud, two-handed, 2 weight, 4\money)
\item Shotgun (3-harm, close, semi-auto, loud, messy, two-handed, 2 weight, 3\money)
\item Rocket Launcher (4-harm, close, far, single, loud, messy, area, 3 weight, 4\money)
\end{itemize}

\subsection{Armour Tags}
\begin{itemize}
\item \textit{$n$ armour}: Damage recieved while wearing this armour is reduced by $n$
%\item \textit{}: 
\end{itemize}