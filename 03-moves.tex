\section{Moves}
The moves are the blood of a *-World game, and they keep the game,
well, moving. Any move consists of a trigger, and an action. A success
generally means the player gets their desired outcome, a partial
success is still a success, but a slightly worse overall outcome, and
a failure means the GM can make a move.

Class- and Mecha-specific moves will be treated in their own sections,
so the basic moves accessible to everyone (human or giant robot) will
be detailed here.

\subsection{Basic Moves}

\movenofail{Brawl}{trade blows or blades with someone}{Hard}
{deal your damage to the enemy and choose one:
\begin{itemize}
\item Deal an extra 1-harm or force the enemy into a nearby danger
\item Avoid a counterattack
\end{itemize}}
{deal your damage to the enemy but receive a counterattack}

Brawl is for dealing with ruffians in melee when diplomatic negotions
break down. There's no trickery or deceit here; your opponent is fully
aware of your attack, and may the strongest fighter win.

\paragraph{Example}
\begin{dialogue}
  \speak{GM} Hiroshi, at your words the guard silently removes his
  sunglasses, puts them in his pocket, and gets down into a fighting
  pose. It looks like he's not going to let you pass unless it's over
  his dead body.

  \speak{Hiroshi} That can be arranged. I launch a haymaker at him.

  \speak{GM} Sounds like a brawl to me. Roll+Hard.

  \speak{Hiroshi} 11. Damn straight. I'll avoid the counterattack.

  \speak{GM} Sure. As you rush towards the guard he aims a fist at
  you, but it whizzes past your ear as your punch smacks him in the
  face. He staggers backwards, surprised.

  \speak{Hiroshi} ``Are you going to let me pass now, or do I have to
  give you another lesson?''

  \speak{GM} ``Hmm, not bad,'' the guard replies, as he retrieves the
  stun baton off his belt, ``but it'll take more than that to beat
  me!'' He gives it a flick and it extends, crackling with
  electricity. What do you do?
\end{dialogue}

\movenofail{Riposte}{anticipate a chance to counterattack}{Fast}
{deal your damage to the enemy and avoid their attack}
{choose one:
\begin{itemize}
\item Deal damage to your enemy
\item Avoid their attack
\end{itemize}}

A Riposte is a more calculated melee move, which involves anticipating
your opponent's attack. In this way you can more easily avoid damage,
but you're not as dangerous yourself. A riposte triggers whenever
you're fighting defensively.

\paragraph{Example}
\begin{dialogue}
  \speak{GM} Yuki, you hear a voice from behind you. ``Here you are,
  my little kitten. Strayed so far from your friends? Why don't you
  play with me for a while?''
  
  \speak{Yuki} I only know one guy who spouts such tired lines! I spin
  around to face him.

  \speak{GM} He just likes to be theatrical, okay? As you guessed,
  you're faced by Captain Flintlock, leader of the Imperial Invasion
  Cohort. He's got his rapier drawn and is advancing towards you.

  \speak{Yuki} This guy gives me the creeps. I'm going to deflect his
  attack with my antique katana and run off.

  \speak{GM} A Riposte? Roll+Fast, then.

  \speak{Yuki} I'm not sure, I'm not really attacking him, just making
  my escape.

  \speak{GM} Hmm, if you're not counterattacking, then it's not really
  a Riposte. Let's make it a Defy on Fast to get your defense up in
  time and flee before he gets you.
\end{dialogue}

\move{Burst}{spray and pray with an semi- or fully-automatic weapon}{Cool}
{deal your damage to the enemy}
{choose two:
\begin{itemize}
\item Deal damage to your enemy
\item Don't hit any unintended targets
\item Still have ammo in the gun
\end{itemize}}
{choose one}

You don't need to be a better shot, you just need to fire more
bullets! Full-auto weapons are hard to control, and unless you're
military trained, which most classes aren't, this is how you'll be
racking up that ammo bill with them. It also goes for semi-auto
weapons, if you feel like hurting everything in the room except you.

\paragraph{Example}
\begin{dialogue}
  \speak{GM} Bruno, as you whizz round the corner, you suddenly come
  face-to-face with a small platoon of guards. They seem as surprised
  to see you as you are at them.

  \speak{Bruno} Shit! I give them a taste of my assault rifle fire.

  \speak{GM} Indiscriminate bullets solve everything. Roll Burst with
  Cool, if you don't mind.

  \speak{Bruno} Uhh, a 5.

  \speak{GM} How cool of you. Right, either you empty your clip,
  leaving yourself vulnerable, or you hit an explosive container that
  will collapse this corridor, blocking passage. Or you can hit them,
  but both of those will happen.

  \speak{Bruno} Will that explosion take out the bad guys?

  \speak{GM} I'm afraid your ``quick thinking'' didn't afford you time to check.

  \speak{Bruno} Whoops.
\end{dialogue}

\movenofail{Aim}{take aim and fire with a semi-auto or single-action weapon}{Plan}
{deal your damage to the enemy}
{deal your damage but choose one:
\begin{itemize}
\item You take what you can get: $-1$-harm
\item You have to displace to get a good shot
\item You have to empty your clip or throw another one
\end{itemize}}

Aiming is a more conservative method of fighing at range, as well as
how you use single-action and thrown weapons. Note this move is
triggered even if the enemy is unaware of you. Aiming is \emph{hard},
you know.

\paragraph{Example}
\begin{dialogue}
  \speak{GM} Alright, Agent 0, the chance you've been waiting for is
  here. You see through your scope the Yakuza boss you've been tailing
  getting out of his armoured limo.

  \speak{Agent 0} You screwed up for the last time. I take the shot.

  \speak{GM} Sounds like Aim to me.

  \speak{Agent 0} 8. Well, I guess I don't need my ammo on this
  distant rooftop. I'll empty the clip.

  \speak{GM} Fair enough. The gangsters begin to scatter as the sound
  of your rifle rings out. Your last shot meets its mark, splattering
  his brains across the alley, but as you scan the carnage, you notice
  your contact, Kensuke, was also there, and you just shot him in the
  leg with a stray bullet.

  \speak{Agent 0} Oh, that's not good.
\end{dialogue}

\movenofail{Hold Fast}{stand your ground}{Cool} % alternate name: Suffer
{you maintain the status quo despite outside forces}
{the GM will offer you a choice. You can maintain the status quo, but you will have to suffer attrition or give up something important. The GM will say what.}

%% You Hold Fast when something is trying to make you give up. When a
%% negotiation starts to go badly, when your character endures something
%% that doesn't obviously just cause harm, or even when you stand in defense of
%% something, you might Hold Fast.

%% On a 7--9, ``attrition'' and ``something important'' can be
%% interpreted broadly. It could be extra cash to smooth over
%% negotiations, straight up harm from lethal radiation, important
%% information like your secret identity, or any number of other things.

%% \movenofail{Persevere}{do something under fire, or dig in to endure fire}{Cool}{you do it.}{you flinch, hesitate, or stall: the GM can offer you a worse outcome, a hard bargain, or an ugly choice.} 

%% \paragraph{Example}
%% \begin{dialogue}
%%   \speak{GM}
%% \end{dialogue}

%% Fast Talk

\movenofail{Taunt}{insult or make a snarky comment at someone}{Talk}
{choose two:
\begin{itemize}
\item you provoke them into making an error
\item you provoke them into revealing information
\item you don't provoke a retaliation
\end{itemize}}
{choose one}

\movenofail{Share Memories}{consult your accumulated knowledge about something}{Plan}{the GM will tell you something interesting and useful about the subject relevant to your situation.}{the GM will only tell you something interesting---it's on you to make it useful.}

The GM might ask you ``How do you know this?'' Tell them the truth, now.

\movenofail{Perceive Truths}{closely study a situation or person}{Plan}{ask the GM 3 questions from the list below.}{ask 1.}

Either way, take $+1$ forward when acting on the answers.

\begin{itemize}
\item What happened here recently?
\item What is about to happen?
\item What should I be on the lookout for?
\item What here is useful or valuable to me?
\item Who's really in control here?
\item What here is not what it appears to be?
\end{itemize}

\movedesc{Defy}{make a last-ditch attempt to avoid a calamity}{pick the stat you are using.}
If you do it:
\begin{itemize}
\item By ignoring the danger or overcoming your fears, use Cool
\item With quick talking, an inspiring word, or a witty retort, use Talk
\item With a feat of strength or endurance, use Hard
\item With lightning reflexes or speed, use Fast
\item By using gadgets or technical knowledge, use Tech
\item With a premeditated strategy or a bright idea, use Plan
\end{itemize}
Assuming you have time to react, Roll+Stat:
\rollresnofail{you do what you set out to, the threat doesn’t come to bear}
{you stumble, hesitate, or flinch: the GM will offer you a worse outcome, hard bargain, or ugly choice}

\movenofail{Jury Rig}{use a piece of equipment for something other than its intended purpose}{Tech}
{you succeed}
{you succeed, but choose one:
\begin{itemize}
\item you damage the equipment beyond repair
\item you set off a previously-unknown function of the equipment
\item you hurt yourself in the process
\end{itemize}}

\subsection{Special Moves}

\movedesc{Intermission}{settle in to rest and maintain your mecha}{restore all depleted Mecha Items. You may also change the type of one of them. If you leveled up, take care of that. When you wake from at least a few uninterrupted hours of sleep heal damage equal to half your max HP.}

\movedesc{Level Up}{mark your 5th experience point}{increment your level and reset your experience to zero. The next Intermission, choose a new advanced move from your class.}

\movedesc{Encumbrance}{make a move while carrying weight}{you may be encumbered. If your weight carried is:}
\begin{itemize}
\item Equal to or less than your load, you suffer no penalty
\item Less than or equal to your load$+2$, you take $-1$ ongoing until you lighten your burden
\item Greater than your load$+2$, you have a choice: drop at least 1 weight and roll at $-1$, or automatically fail
\end{itemize}

\move{Source Parts}{browse a store's catalogue for mundane mecha parts, personal items, or repairs}{Talk}
{they have what you need, and you can get a good deal}
{they have what you need, for about what you expected}
{they have what you need, but it costs more than you thought it would}

\movenofail{Rare Parts}{make a request to a store for rare mecha parts}{Talk and describe what you're looking for}
{they happen to have it in, or the nearest thing that exists, and you can have it for the right price}
{they have it, or can get it, but there's a catch. The GM will choose one:
\begin{itemize}
\item It costs a lot.
\item It's not quite what you wanted.
\item It comes with strings attached, or a favour owed.
\end{itemize}}

\movedesc{End Of Session}{reach the end of the session}{answer the following three questions as a group}
\begin{itemize}
\item Did we learn something new and important about the world?
\item Did we overcome a notable enemy?
\item Did we meet an interesting character?
\end{itemize}

For each ``yes'' answer everyone marks experience.

Choose a character who knows you better than they used to. If there's
more than one, choose one at your whim. Tell that player to add 1 to
their Hx with you on their sheet. If this brings them to Hx$+4$, they
reset to Hx$+1$ (and therefore mark experience).
