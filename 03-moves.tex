\section{Moves}
The moves are the blood of a PtbA game, and they keep the game,
well, moving. Any move consists of a trigger, and an action. A success
generally means the player gets their desired outcome, a partial
success is still a success, but a slightly worse overall outcome, and
a miss means the GM can make a move.

Class- and Mecha-specific moves will be treated in their own sections,
so the basic moves accessible to everyone (human or giant robot) will
be detailed here.


\subsection{Basic Moves}


\move{Brawl}{trade blows or blades with someone}{Hard}
{deal your damage to the enemy and choose one:
\begin{itemize}
\item Deal an extra 1-harm
\item Avoid their attack
\end{itemize}}
{deal your damage to the enemy but they also attack you}

Brawl is for dealing with ruffians in melee when diplomatic negotions break down.
Taking someone by surprise or attacking a helpless enemy isn't Brawl; you just deal your damage.

\paragraph{Example}
\begin{dialogue}
  \speak{GM} Hiroshi, at your words the guard silently removes his
  sunglasses, puts them in his pocket, and gets down into a fighting
  pose. It looks like he's not going to let you pass unless it's over
  his dead body.

  \speak{Hiroshi} That can be arranged. I launch a haymaker at him.

  \speak{GM} Sounds like a brawl to me. Roll+Hard.

  \speak{Hiroshi} 11. Damn straight. I'll avoid the counterattack.

  \speak{GM} Sure. As you rush towards the guard he aims a fist at
  you, but it whizzes past your ear as your punch smacks him in the
  face. He staggers backwards, surprised.

  \speak{Hiroshi} ``Are you going to let me pass now, or do I have to
  give you another lesson?''

  \speak{GM} ``Hmm, not bad,'' the guard replies, as he retrieves the
  stun baton off his belt, ``but it'll take more than that to beat
  me!'' He gives it a flick and it extends, crackling with
  electricity. What do you do?
\end{dialogue}


\move{Lookout}{protect an ally, object or location from an enemy}{Fast}
{hold three}
{hold one}
Spend your hold 1-for-1 on the following effects when your protectee comes under fire:
\begin{itemize}
\item Deal damage to the enemy
\item Divert the enemy's attention to yourself
\item Give 1-armour to the attack's target, or retain hold of it
\end{itemize}

%A Riposte is a more calculated melee move, which involves anticipating
%your opponent's attack. In this way you can more easily avoid damage,
%and even intercept attacks not meant for you, but you're not as
%dangerous yourself. A riposte triggers whenever you're fighting
%defensively.
%
%\paragraph{Example}
%\begin{dialogue}
%  \speak{GM} Yuki, you hear a voice from behind you. ``Here you are,
%  my little kitten. Strayed so far from your friends? Why don't you
%  play with me for a while?''
%  
%  \speak{Yuki} I only know one guy who spouts such tired lines! I spin
%  around to face him.
%
%  \speak{GM} He just likes to be theatrical, okay? As you guessed,
%  you're faced by Captain Flintlock, leader of the Imperial Invasion
%  Cohort. He's got his rapier drawn and is advancing towards you.
%
%  \speak{Yuki} This guy gives me the creeps. I'm going to deflect his
%  attack with my antique katana and run off.
%
%  \speak{GM} A Riposte? Roll+Fast, then.
%
%  \speak{Yuki} I'm not sure, I'm not really attacking him, just making
%  my escape.
%
%  \speak{GM} Hmm, if you're not counterattacking, then it's not really
%  a Riposte. Let's make it a Dash to get your defense up in
%  time and flee before he gets you.
%\end{dialogue}


\move{Aim}{take aim and fire with a semi-auto or single-action weapon}{Plan}
{deal your damage to the enemy}
{deal your damage but choose one:
\begin{itemize}
\item You take what you can get: $-1$-harm
\item You have to displace to get a good shot
\item You have to empty your clip or throw another one
\end{itemize}}

Aiming is the most conservative method of fighing at range, as well as
how you use single-action and thrown weapons. Note this move is
triggered even if the enemy is unaware of you. Aiming is \emph{hard},
you know.

\paragraph{Example}
\begin{dialogue}
  \speak{GM} Alright, Agent 0, the chance you've been waiting for is
  here. You see through your scope the Yakuza boss you've been tailing
  getting out of his armoured limo.

  \speak{Agent 0} You screwed up for the last time. I take the shot.

  \speak{GM} Sounds like Aim to me.

  \speak{Agent 0} 8. Well, I guess I don't need my ammo on this
  distant rooftop. I'll empty the clip.

  \speak{GM} Fair enough. The gangsters begin to scatter as the sound
  of your rifle rings out. Your last shot meets its mark, splattering
  his brains across the alley, but as you scan the carnage, you notice
  your contact, Kensuke, was also there, and you just shot him in the
  leg with a stray bullet.

  \speak{Agent 0} Oh, that's not good.
\end{dialogue}


\move{Burst}{spray and pray with a semi- or fully-automatic weapon}{Cool}
{deal your damage to the enemy}
{deal your damage and choose one:
\begin{itemize}
\item Don't hit any unintended targets
\item Still have ammo in the gun
\end{itemize}}

You don't need to be a better shot, you just need to fire more
bullets! Full-auto weapons are hard to control, and unless you're
military trained, which most classes aren't, this is how you'll be
racking up that ammo bill with them. It also goes for semi-auto
weapons, if you feel like hurting everything in the room except you.

\paragraph{Example}
\begin{dialogue}
  \speak{GM} Bruno, as you whizz round the corner, you suddenly come
  face-to-face with a small platoon of guards. They seem as surprised
  to see you as you are at them.

  \speak{Bruno} Shit! I give them a taste of my assault rifle fire.

  \speak{GM} Indiscriminate bullets solve everything. Roll Burst with
  Cool, if you don't mind.

  \speak{Bruno} Uhh, a 5.

  \speak{GM} How cool of you. Right, either you empty your clip,
  leaving yourself vulnerable, or you hit an explosive container that
  will collapse this corridor, blocking passage. Or you can hit them,
  but both of those will happen.

  \speak{Bruno} Will that explosion take out the bad guys?

  \speak{GM} I'm afraid your ``quick thinking'' didn't afford you time to check.

  \speak{Bruno} Whoops.
\end{dialogue}


%% \move{Hold Fast}{stand your ground}{Cool} % alternate name: Suffer
%% {you maintain the status quo despite outside forces}
%% {the GM will offer you a choice. You can maintain the status quo, but you will have to suffer attrition or give up something important. The GM will say what.}

%% You Hold Fast when something is trying to make you give up. When a
%% negotiation starts to go badly, when your character endures something
%% that doesn't obviously just cause harm, or even when you stand in defense of
%% something, you might Hold Fast.

%% On a 7--9, ``attrition'' and ``something important'' can be
%% interpreted broadly. It could be extra cash to smooth over
%% negotiations, straight up harm from lethal radiation, important
%% information like your secret identity, or any number of other things.


\move{Persevere}{do something in a stressful situation}{Cool}{you do it.}{you flinch, hesitate, or stall: the GM can offer you a worse outcome, a hard bargain, or an ugly choice.} 

Ganbare! You must Persevere when the situation really doesn't look good for you. You don't usually have to persevere just to do something in a combat situation, unless the stakes are really high.

%% make it obvious that this is the ``make something more difficult'' move for the GM?

\paragraph{Example}
\begin{dialogue}
  \speak{GM} You rush into the room in which Princess Amelia is
  supposedly held. Looks like your intel was accurate; she's
  here. However, so is the evil Count Draconis, with an arm around her
  neck and a gun to her head. ``Step back, you fools!'' he bellows,
  ``You wouldn't want anything to happen to the young princess here,
  would you?''

  \speak{LeVallier} What a coward! I'll draw my pistol and take him
  out before he has a chance to harm her.

  \speak{GM} Hmm, okay. Well, I'd say you're acting under fire. You
  can't risk harming the princess. Roll+Cool, if you please.

  \speak{LeVallier} Fair enough. 11.

  \speak{GM} Nice. You whip out your pistol and point it straight at him. 
  He looks shocked---he wasn't expecting you to be so brazen, and he
  moves his gun away from the princess to train it on you. Give us an
  Aim roll and let's see how dead he is.
\end{dialogue}


\move{Bargain}{strike a deal with someone}{Talk}{they'll hold up their end of the bargain, and expect you to do the same.}{they want to see you keep your promise first, or at least concrete assurance that you will.}


\move{Fast Talk}{make something up on the spot}{Talk}{everyone believes you.}{you raise suspicion, but people go along with you, for now.}

\paragraph{Example}
\begin{dialogue}
  \speak{GM} Sanya, as you pass through this door, you gently bump
  into a guard coming the other way. ``What are you doing here?'' he
  asks, ``This is a restricted area.'' He has his hand on his rifle.

  \speak{Sanya} ``I was just looking for the toilets, and I somehow
  got lost.'' I giggle and bump my fist to the side of my
  head. ``Tehehe!''

  \speak{GM} Very convincing. Roll+Talk, please.

  \speak{Sanya} 7.

  \speak{GM} ``Is that so?'' He has his suspicions. ``In that case,
  I'll escort you back to the visitors' area. Please come with me.''
  He grabs your shoulder roughly and starts walking.

  \speak{Sanya} Tsk. Now I've got to shake this guy somehow\ldots
\end{dialogue}

\move{Dash}{try to evade a danger or get somewhere quickly}{Fast}{you're fast enough.}{you make it, but your hesitation costs you. You leave something behind, or end up in a vulnerable spot, GM's choice.}

\move{Flashback}{consult your accumulated knowledge about something}{Plan}{the GM will tell you something interesting and useful about the subject relevant to your situation.}{the GM will only tell you something interesting---it's on you to make it useful.}

The GM might ask you ``How do you know this?'' Tell them the truth, now.

%\move{Connect}{try to figure someone out}{Talk}{ask their player 3 questions from the following list.}{ask 1.}
%\begin{itemize}
%\item 
%\end{itemize}

\move{Situational Analysis}{stop to take in or investigate your surroundings}{Plan}{ask the GM 3 questions from the list below.}{ask 1.}

Either way, take $+1$ forward when acting on the answers.

\begin{itemize}
\item What happened here recently?
\item What is about to happen?
\item What should I be on the lookout for?
\item What here is useful or valuable to me?
\item Who's really in control here?
\item What here is not what it appears to be?
\end{itemize}

%% \movedesc{Defy}{make a last-ditch attempt to avoid a calamity}{pick the stat you are using.}
%% If you do it:
%% \begin{itemize}
%% \item By ignoring the danger or overcoming your fears, use Cool
%% \item With quick talking, an inspiring word, or a witty retort, use Talk
%% \item With a feat of strength or endurance, use Hard
%% \item With lightning reflexes or speed, use Fast
%% \item By using gadgets or technical knowledge, use Tech
%% \item With a premeditated strategy or a bright idea, use Plan
%% \end{itemize}
%% Assuming you have time to react, Roll+Stat:
%% \rollresnofail{you do what you set out to, the threat doesn’t come to bear}
%% {you stumble, hesitate, or flinch: the GM will offer you a worse outcome, hard bargain, or ugly choice}

% Aid or Interfere

%% Missing Moves: Seduce or Manipulate/Parley


\subsection{Special Moves}


\movedesc{Intermission}{settle in to rest and maintain your mecha}{restore all depleted Mecha Items. You may also change the type of one of them. If you leveled up, take care of that. When you wake from at least a few uninterrupted hours of sleep heal damage equal to half your max HP.}


% Change to more AW-like leveling
%\movedesc{Level Up}{mark your 5th experience point}{increment your level and reset your experience to zero. The next Intermission, choose a new advanced move from your class.}

%\movedesc{Encumbrance}{make a move while carrying weight}{you may be encumbered. If your weight carried is:}
%\begin{itemize}
%\item Equal to or less than your load, you suffer no penalty
%\item Less than or equal to your load$+2$, you take $-1$ ongoing until you lighten your burden
%\item Greater than your load$+2$, you have a choice: drop at least 1 weight and roll at $-1$, or automatically fail
%\end{itemize}


%% rolling for this isn't interesting, and opens the gates for xp grinding
\movedesc{Source Parts}{browse a store's catalogue for mundane mecha parts, personal items, or repairs}{you can get it for market price.}


\move{Rare Parts}{make a request to a store for rare mecha parts}{Talk and describe what you're looking for}
{they happen to have it in, or the nearest thing that exists, and you can have it for the right price}
{they have it, or can get it, but there's a catch. The GM will choose one:
\begin{itemize}
\item It costs a lot.
\item It's not quite what you wanted.
\item It comes with strings attached, or a favour owed.
\end{itemize}}


\movedesc{End Of Session}{reach the end of the session}{answer the following three questions as a group}
\begin{itemize}
\item Did we learn something new and important about the world?
\item Did we overcome a notable enemy?
\item Did we meet an interesting character?
\end{itemize}

For each ``yes'' answer everyone marks experience.

Choose a character who knows you better than they used to. If there's
more than one, choose one at your whim. Tell that player to add 1 to
their Hx with you on their sheet. If this brings them to Hx$+4$, they
reset to Hx$+1$ (and therefore mark experience).
