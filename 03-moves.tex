\section{Moves}
The moves are the blood of a *-World game, and they keep the game,
well, moving. Any move consists of a trigger, and an action. A success
generally means the player gets their desired outcome, a partial
success is still a success, but a slightly worse overall outcome, and
a failure means the GM can make a move.

Class- and Mecha-specific moves will be treated in their own sections,
so the basic moves accessible to everyone (human or giant robot) will
be detailed here.

\subsection{Basic Moves}

\movenofail{Brawl}{trade blows or blades with someone}{Hard}
{deal your damage to the enemy and choose one:
\begin{itemize}
\item Force the enemy into a nearby danger
\item Avoid a counterattack
\end{itemize}}
{deal your damage to the enemy but receive a counterattack}

\movenofail{Flourish}{execute a fancy combat technique}{Fast}
{deal your damage to the enemy and avoid their attack}
{choose one:
\begin{itemize}
\item Deal damage to your enemy
\item Avoid a counterattack
\end{itemize}}

\move{Burst}{spray and pray with an semi- or fully-automatic weapon}{Cool}
{deal your damage to the enemy}
{choose two:
\begin{itemize}
\item Deal damage to your enemy
\item Don't hit any unintended targets
\item Still have ammo in the gun
\end{itemize}}
{choose one}

\movenofail{Aim}{take aim and fire with a semi-auto or single-action weapon}{Plan}
{deal your damage to the enemy}
{deal your damage but choose one:
\begin{itemize}
\item You take what you can get: $-1$-harm
\item You have to displace to get a good shot
\item You have to empty your clip or throw another one
\end{itemize}}

\move{Hold Fast}{stand your ground}{Cool} % alternate name: Suffer
{you maintain the status quo despite outside forces}
{choose two:
\begin{itemize}
\item maintain the status quo
\item you don't take attrition
\item you don't lose something important
\end{itemize}}
{choose one}

\movenofail{Taunt}{insult or make a snarky comment at someone}{Talk}
{choose two:
\begin{itemize}
\item you provoke them into making an error
\item you provoke them into revealing information
\item you don't provoke a retaliation
\end{itemize}}
{choose one}

\movenofail{Share Memories}{consult your accumulated knowledge about something}{Plan}{the GM will tell you something interesting and useful about the subject relevant to your situation.}{the GM will only tell you something interesting---it's on you to make it useful.}

The GM might ask you ``How do you know this?'' Tell them the truth, now.

\movenofail{Perceive Truths}{closely study a situation or person}{Plan}{ask the GM 3 questions from the list below.}{ask 1.}

Either way, take $+1$ forward when acting on the answers.

\begin{itemize}
\item What happened here recently?
\item What is about to happen?
\item What should I be on the lookout for?
\item What here is useful or valuable to me?
\item Who's really in control here?
\item What here is not what it appears to be?
\end{itemize}

\movedesc{Defy}{make a last-ditch attempt to avoid a calamity}{pick the stat you are using.}
If you do it:
\begin{itemize}
\item By ignoring the danger or overcoming your fears, use Cool
\item With quick talking, an inspiring word, or a witty retort, use Talk
\item With a feat of strength or endurance, use Hard
\item With lightning reflexes or speed, use Fast
\item By using gadgets or technical knowledge, use Tech
\item With a premeditated strategy or a bright idea, use Plan
\end{itemize}
Assuming you have time to react, Roll+Stat:
\rollresnofail{you do what you set out to, the threat doesn’t come to bear}
{you stumble, hesitate, or flinch: the GM will offer you a worse outcome, hard bargain, or ugly choice}

\movenofail{Jury Rig}{use a piece of equipment for something other than its intended purpose}{Tech}
{you succeed}
{you succeed, but choose one:
\begin{itemize}
\item you damage the equipment beyond repair
\item you set off a previously-unknown function of the equipment
\item you hurt yourself in the process
\end{itemize}}

\subsection{Special Moves}

\movedesc{Intermission}{settle in to rest and maintain your mecha}{restore all depleted Mecha Items. If you leveled up, take care of that. When you wake from at least a few uninterrupted hours of sleep heal damage equal to half your max HP.}

\movedesc{Level Up}{mark your 5th experience point}{increment your level and reset your experience to zero. The next Intermission, choose a new advanced move from your class.}

\movedesc{Encumbrance}{make a move while carrying weight}{you may be encumbered. If your weight carried is:}
\begin{itemize}
\item Equal to or less than your load, you suffer no penalty
\item Less than or equal to your load$+2$, you take $-1$ ongoing until you lighten your burden
\item Greater than your load$+2$, you have a choice: drop at least 1 weight and roll at $-1$, or automatically fail
\end{itemize}

\move{Source Parts}{browse a store's catalogue for mundane mecha parts, personal items, or repairs}{Talk}
{they have what you need, and you can get a good deal}
{they have what you need, for about what you expected}
{they have what you need, but it costs more than you thought it would}

\movenofail{Rare Parts}{make a request to a store for rare mecha parts}{Talk and describe what you're looking for}
{they happen to have it in, or the nearest thing that exists, and you can have it for the right price}
{they have it, or can get it, but there's a catch. The GM will choose one:
\begin{itemize}
\item It costs a lot.
\item It's not quite what you wanted.
\item It comes with strings attached, or a favour owed.
\end{itemize}}

\movedesc{End Of Session}{reach the end of the session}{answer the following three questions as a group}
\begin{itemize}
\item Did we learn something new and important about the world?
\item Did we overcome a notable enemy?
\item Did we meet an interesting character?
\end{itemize}

For each ``yes'' answer everyone marks experience.

Choose a character who knows you better than they used to. If there's
more than one, choose one at your whim. Tell that player to add 1 to
their Hx with you on their sheet. If this brings them to Hx$+4$, they
reset to Hx$+1$ (and therefore mark experience).
